		\documentclass{report}
		
		% Info
		\title{Cinematic storytelling in VR}
		\author{Wouter Vanmulken}
		\date{December 2017}
		
		% Packages 
		\usepackage{graphicx}
		\usepackage{hyperref}
		
		%Titles
		\usepackage[T1]{fontenc}
		\usepackage{titlesec, blindtext, color}

		\definecolor{gray75}{gray}{0.75}
		\newcommand{\hsp}{\hspace{20pt}}
		\titleformat{\chapter}[hang]{\Huge\bfseries}{\thechapter\hsp\textcolor{gray75}{|}\hsp}{0pt}{\Huge\bfseries}
		\setlength{\parindent}{0em}
		\setlength{\parskip}{1em}
		\usepackage[a4paper, total={6in, 8in}]{geometry}
		
		\usepackage{todonotes}
		
		\begin{document}
				\pagenumbering{gobble}
				\maketitle
				\tableofcontents
				\newpage
				\pagenumbering{arabic}
		
				
				\chapter{Research questions}
		
				\begin{itemize}
					\item How can we make people feel present in a story ?
					\item How can we keep people from missing key story details ?
					\item How can we grab the attention of a user ?
				\end{itemize}
				\begin{par}
					All of my research boils down to one single question. And that is how to use composition in VR like they do in cinema. Therefore this will be the main objective of my research.
					\todo{write more here}
				\end{par}

			

				\chapter{How can we make people feel present in a story ?}
								
				Making people feel present is a delicate balance, a world that doesn't feel right can be a big problem in a storytelling experience. A real-world example of this would be, someone talking during a movie. This often brings people back to reality and can be quite frustrating. Although this might be annoying when watching a movie, it can be quite devastating while in VR. Breaking the illusion of a virtual world even once can make it feel unreal or even more important, uninteresting.\todo{uninteresting the best thing ?}

				Currently there isn't a lot of official research on what these things might be in relation to cinematic experiences. But luckily there still is Oculus-storytellingstudio. Oculus-storytellingstudio was a department of Oculus specifically set up to test how we can make cinematic experiences in VR. To do this they have been making the first steps into animating storytelling experiences. With the motto of "story is king", they are currently \textbf{the} resource on storytelling in VR. Unfortunately the department has been closed because they didn't want to compete in the marketplace they were developing for. But not before they put a lot of the research they did online and making some amazing proof-of-concept pieces like Henry and Dear Angelica.
				
				
				To share problems and solutions they found, they set up a blog. And a very interesting one at that. The only way to solve these problem is to make experiences but unfortunately this will be out of scope for me personally. Because of that i'll talk about a couple of main issues they found and how to prevent them.
				
				\begin{itemize}
					\item Not being acknowledge in the world or the .
					\item Only one thing happening at a time in the world.
					\item Walking into objects.
				\end{itemize}
				
				
		\end{document}