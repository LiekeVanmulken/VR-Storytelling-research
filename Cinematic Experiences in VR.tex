		\documentclass{report}
		
		% Info
		\title{Cinematic storytelling in VR}
		\author{Wouter Vanmulken}
		\date{December 2017}
		
		% Packages 
		\usepackage{graphicx}
		\usepackage{hyperref}
		
		%Titles
		\usepackage[T1]{fontenc}
		\usepackage{titlesec, blindtext, color}

		\definecolor{gray75}{gray}{0.75}
		\newcommand{\hsp}{\hspace{20pt}}
		\titleformat{\chapter}[hang]{\Huge\bfseries}{\thechapter\hsp\textcolor{gray75}{|}\hsp}{0pt}{\Huge\bfseries}
		\setlength{\parindent}{0em}
		\setlength{\parskip}{1em}
		\usepackage[a4paper, total={6in, 8in}]{geometry}
		\usepackage[scaled]{helvet}
		\renewcommand\familydefault{\sfdefault} 
		\usepackage[T1]{fontenc}
		
		\usepackage{todonotes}
		\usepackage{gensymb}
		\usepackage{graphicx}
		
		\begin{document}
				\pagenumbering{gobble}
				\maketitle
				\tableofcontents
				\newpage
				\pagenumbering{arabic}
		
				\chapter{Introduction}
				
				This is a research paper about storytelling in VR. As there is still a lot of information undiscovered about storytelling in VR, i will also be talking about my personal observations on VR storytelling experiences. Therefore this document will mostly contain opinions on the current state and possibilities and should not be used as empirical evidence.\todo{maybe not rek your crdibility immediately}
				
				\chapter{Research questions}
		
				\begin{itemize}
					\item How can we make people feel present in a story?
					\item How can we keep people from missing key story details?
					\item How can we grab the attention of a user?
				\end{itemize}
				\begin{par}
					All of my research boils down to one single question. And that is how to use composition in VR like they do in cinema. Therefore this will be the main objective of my research.
					\todo{write more here}
				\end{par}

				
				\chapter{How can we make people feel present in a story ?}
								
				Making people feel present is a delicate balance, a world that doesn't feel right can be a big problem in a storytelling experience. A real-world example of this would be, someone talking during a movie. This often brings people back to reality and can be quite frustrating. Although this might be annoying when watching a movie, it can be quite devastating while in VR. Breaking the illusion of a virtual world even once can make it feel unreal or even more important, uninteresting.\todo{uninteresting the best thing ?}

				Currently there isn't a lot of official research on what these things might be in relation to cinematic experiences. But luckily there still is Oculus-storytellingstudio. Storytellingstudio was a department of Oculus specifically set up to test how cinematic experiences in VR can be created. To be able to do this they have been making the first steps into animating storytelling experiences. With the motto of "story is king", they are currently still \textbf{the} resource on storytelling in VR. 
				
				Unfortunately the department has been closed since they didn't want to compete in a marketplace they were creating research for. But not before they put a lot of the research online and making some amazing proof-of-concept pieces like Henry and Dear Angelica.
				
				
				To share problems and solutions they found, they set up a blog. And a very interesting one at that. The only way to solve these problem is to make experiences but unfortunately this will be out of scope for me personally. Because of that i'll talk about a couple of main issues they found and how to prevent them.
				
				\begin{itemize}
					\item Not being acknowledge in the world or the .
					\item Only one thing happening at a time in the world.
					\item Walking into objects.
				\end{itemize}
				
				
				
				\chapter{How can we keep people from miss key story details ?}
				
				You probably noticed that the title of the chapter is missing an "ing". This is just an example of how a users would feel, missing a part of the story in VR. Missing information can get quite frustrating as you can imagine, in this instance you probably blamed me and you would be right to do so. In a VR experience the same thing would happen. The user thinks you didn't program things properly and they start to doubt your experience as well as feel stupid that they didn't notice the information.
				
				\chapter{How can we grab the attention of a user ?}
				\todo{maybe add an image that grabs the attention}
				
				\section{Should we be guiding the attention}
				Before we can even answer the question of how we can grab the attention we need to answer the question of if we should. VR gives us 360$^{\circ}$ vision of a alternate reality, so should we limit ourselves because we aren't used to it yet. When the first movie was projected, which was of a train coming into the station, people wanted to jump out of their seats \todo{add source "der spiegel" for this} but with time we've overcome this. These days 3D movies don't even make us flinch when something is coming at you. The same thing happens when people enters VR for the first time. If you fire a projectile at them, they will move away from it and if you make them fall they'll try to cushion their impact by bending their knees. Currently we are still in the stage were people keep jumping out in front of the train. But looking at how cinema progressed, VR might as well. 
				
				So keeping this example in mind, should we be catering purely to people who want to jump away from the train or should we be finding new ways to make people comfortable with standing in front of that train.
				
				\section{Limiting attention space to 180$^{\circ}$}
				 \todo{make a smooth transition}The team that made Henry said in a presentation that this is currently still a problem that they don't have a solution for. They found that for now they were still doing experiences in 180 degrees and would in the future like to find a better solution to this. Which they partly did in Dear angelica wherein they used 360 degree of the users environment to draw what can only be described as a VR comic book being drawn while you look. In my opinion this had still a major disadvantage and that it lost me a couple of times. Which made me feel like a absolute idiot and like i was missing a part of the story, which frustrated me endlessly. That being said it was an amazing experiment in 360$^{\circ}$ is storytelling.
				
				\section{Tools to use}
			
				Now that we have looked at what things we might and might not want to do, we can look at tools that can be used to focus a users attention.
				
				\subsection{Binaural audio}
				Binaural audio is one of the most intrusive as well as effective tools in your toolbox. Which is probably also why its the first one everyone comes up with. It's effective because we see sudden and unexpected sound as something to investigate. Because of this it's something that catches our attention immediately.			
				This might very well\todo{maybe a source for this} be an old reflex of when there were still predators that could might be dangerous. \todo{this needs a source or experience} 
				
				This also means that we need a source of that sound to make sure it's not a threat. If you make a sound and don't explain it, people will get uncomfortable. \todo{give a example of henry and the music}
				
				\subsection{Pattern interrupt}
				test
				\begin{figure}
 					\includegraphics[width=25em]{img/PatternInterrupt.jpg}
					\caption{A example of pattern interrupt}
					\label{fig:PatternInterrupt1}
				\end{figure}
				
				Figure \ref{fig:PatternInterrupt1} shows an example of a repetitive pattern that is interrupted by something that doesn't match the pattern.
				
				%phantom sound https://www.oculus.com/story-studio/blog/binaural-audio-for-narrative-vr/
				\subsubsection{music and Narration}
				
				
				
				\subsection{Negative space}
				\subsection{Guiding the attention}
				
				\chapter{Conclusion}
				\section{Where is VR storytelling currently}
				
				\section{Personal opinion}
				
				\section{The future of VR storytelling}
				
				\chapter{Sources}
				
				image pattern interrupt : \href{https://dealerwebb.com/WebSites/1626/Images/Blogs/2241/PatternInterrupt.jpg}{https://dealerwebb.com/WebSites/1626/Images/Blogs/2241/PatternInterrupt.jpg}
				
		\end{document}