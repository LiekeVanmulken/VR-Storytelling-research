		\documentclass{report}
		
		% Info
		\title{Cinematic storytelling in VR}
		\author{Wouter Vanmulken}
		\date{December 2017}
		
		% Packages 
		\usepackage{graphicx}
		\usepackage{hyperref}
		
		%Titles
		\usepackage[T1]{fontenc}
		\usepackage{titlesec, blindtext, color}
		\definecolor{gray75}{gray}{0.75}
		\newcommand{\hsp}{\hspace{20pt}}
		\titleformat{\chapter}[hang]{\Huge\bfseries}{\thechapter\hsp\textcolor{gray75}{|}\hsp}{0pt}{\Huge\bfseries}
		
		\usepackage{todonotes}
		
		\begin{document}
				\pagenumbering{gobble}
				\maketitle
				\tableofcontents
				\newpage
				\pagenumbering{arabic}
		
				
				\chapter{Research questions}
		
				\begin{itemize}
					\item How can we make people feel present in a story ?
					\item How can we keep people from missing key story details ?
					\item How can we grab the attention of a user ?
				\end{itemize}
				\begin{par}
					All of my research boils down to one single question. And that is how to use composition in VR like they do in cinema. Therefore this will be the main objective of my research.
				\end{par}

			

				\chapter{How can we make people feel present in a story ?}
								
				Making people feel present is a delicate balance, a world that doesn't feel right can be a big problem in a storytelling experience. A real-world example of this would be someone talking during a movie. This often brings people back to reality and can be quite frustrating. Although this might be annoying when watching a movie, it can be quite devastating while in VR. Breaking the illusion of the virtual world even once can make it feel unreal or even more important, uninteresting.

				%maybe remove this
				 To make sure the user don't experience anything that might pull them back to reality the product needs to be tested, tested some more and then tested again.
				
				Currently there isn't a lot of official research on what these things might be in relation to cinematic experiences. But luckily there still is Oculus-studio. Oculus-studio has been making the first steps into animating new storytelling experiences. With the motto of "story is king" which they got from Pixar, they are \textbf{the} resource on storytelling in VR. 
				
				To jumpstart storytelling in VR, Oculus started a team that made experiences with the  
				
				 has a very extensive blog about the things they noticed while making stories. There are some things that will make a world feel off and break the illusion of the world.
				\begin{itemize}
					\item Not being acknowledge in the world.
					\item Only one thing happening at a time in the world.
					\todo{add a list item}
				\end{itemize}
				
				
		\end{document}